\documentclass{article}


\usepackage{amsmath}
\usepackage{amsthm}


\newcommand{\complexfield}{\mathcal{C}}


\newtheorem{prop}{Proposition}


\title{Calculation sheet for Math Apptitude Test}
\date{}
\author{Takumi Enomoto}

\begin{document}
\maketitle

\begin{enumerate}

  \item
    Now $\frac{df}{dx} = \exp(-x) - x \exp(-x) = \exp(-x) - f$,
    so
    $$
      \frac{d^2f}{dx^2} = -\exp(-x) - \frac{df}{dx} = -2 \exp(-x) + f.
    $$

  \item 
    For $y = \ln|1-x|$, $\frac{df}{dx} = -\frac{1}{1-x}$.
    Hence $\frac{d^2f}{dx^2} = -\frac{1}{(1-x)^2}$ and
    $$
    \frac{df}{dx} + x \frac{d^2f}{dx^2} = -\frac{1}{1-x} - \frac{x}{(1-x)^2} = - \frac{1}{(1-x)^2}.
    $$

  \item 
    Now $\frac{(x+h)^3 - x^3}{h} = 3 x^2 + 3xh + h^2$, so
    $$
      \lim_{h \to 0} \frac{(x+h)^3 - x^3}{h} = 3 x^2.
    $$

  \item 
    If $x$ is near $0$ and $x \neq 0$, $\frac{2x + \sin x}{x(x-1)} = \frac{2 + \frac{\sin x}{x}}{x-1}$.
    Using $\lim_{x \to 0} \frac{\sin x}{x}  = 1$,
    $$
      \lim_{x \to 0} \frac{2x + \sin x}{x(x-1)} = \lim_{x \to 0} \frac{2 + \frac{\sin x}{x}}{x-1} = \frac{2 + 1}{0 - 1} = -3.
    $$

  \item
    Using a variable transformation $x^2 = u$ in the domain $x \geq 0$,
    $2x dx = du$ and  
    $$
      I = \int_{0}^{\infty} x \exp(-x^2) dx
        = \frac{1}{2} \int_{0}^{\infty} \exp(-u) du = \frac{1}{2}.
    $$

  \item 
    Using $\frac{1}{x^2 + 3x + 2} = \frac{1}{x+1} - \frac{1}{x+2}$,
    \begin{align*}
      \int_{0}^{1} \frac{1}{x^2 + 3x + 2} dx &= \int_{0}^{1} \left( \frac{1}{x+1} - \frac{1}{x+2} \right) dx \\
       &= \bigg[ \ln (x+1) - \ln(x+2) \bigg]_{0}^{1} \\
       &= ( \ln 2 - \ln 3 ) - ( \ln 1 - \ln 2) = - \ln \frac{3}{4}.
    \end{align*}

  \item 
    The $n$-th order derivative of $f(x) = \exp x$ is $\exp x$, so the $n$-th order term
    in the Taylor series expansion of $f$  about $x=-3$ is
    $$
        \frac{e^{-3}}{n!} (x - (-3))^n.
    $$

  \item
    The derivative of the function $\ln (1+x) = \frac{1}{1+x}$, so
    in some convergence radius,
    \begin{align*}
      \ln (1+x) &= \int \frac{1}{1+x} dx = \int dx \sum_{n=0}^{\infty} (-1)^n x^n \\
        &= \sum_{n=0}^{\infty} \int dx (-1)^n x^n \\
        &= \sum_{n=0}^{\infty} \frac{1}{n+1} (-1)^n x^{n+1}.
    \end{align*}

  \item 
    $k$ needs to satisfy $\int_{-\infty}^{\infty} p(x) dx = 1$. Now
    $$
      \int_{-\infty}^{\infty} p(x) dx = \int_{0}^{1} k (1-x^3) dx
      = k \bigg[ x - \frac{1}{4} x^4 \bigg]_{0}^{1}  = \frac{3}{4} k,
    $$
    so $k = \frac{4}{3}$.

  \item 
    For given random variable $X$, the probablity $P(0 \leq X \leq \frac{1}{2})$ is
    $$
      P(0 \leq X \leq \frac{1}{2}) = \int_{0}^{\frac{1}{2}} \frac{3}{4} (1-x^2) dx
        = \frac{3}{4} \bigg[ x - \frac{1}{3} x^3 \bigg]_{0}^{\frac{1}{2}} = \frac{11}{32}. 
    $$

  \item
    $E(X^2) = E((X - \mu)^2) + \mu^2 = \sigma^2 + \mu^2 = 2^2 + 4^2 = 20$.
  
  \item 
    The skew of the distribution with the mean $\mu$ and the standard derivation $\sigma$ is
    $\frac{ E((X-\mu)^3) } {\sigma ^3}$.
    The uniform distribution is symmetric around its mean, so the integral $\int_{0}^{1} (x-\frac{1}{2})^3 dx = 0 $ and our skew is $0$.

  \item
    First we calculate $V = E((X-\mu)^2)$. Set $s = \frac{\sqrt{\sigma}}{\gamma}$, then
    \begin{align*}
      V &= 
        \int_{-\infty}^{\infty}
          \frac{1}{\sqrt{2\pi s^2}}
          x \cdot
          x \exp \left( - \frac{(x- \mu)^2}{2 s^2} \right)
          dx \\
      &= \frac{1}{\sqrt{2\pi s^2}}
        \left\{ \bigg[
          x \cdot
          \left( -s^2 \right)
          \exp \left( - \frac{x^2}{2 s^2} \right)
          \bigg]_{-\infty}^{\infty}
          - \int_{-\infty}^{\infty}
          \left( -s^2 \right)
          \exp \left( - \frac{x^2}{2 s^2} \right)
          dx
        \right\} \\
      &= \frac{1}{\sqrt{2\pi s^2}}
        \cdot s^2
        \int_{-\infty}^{\infty}
        \exp \left( - \frac{x^2}{2 s^2} \right)
        dx \\
      &= \frac{1}{\sqrt{2 \pi s^2}}
        \cdot s^2
        \cdot \sqrt{2 \pi s^2} = s^2.
    \end{align*}
    Hence the standard derivation which we calculate is $s = \frac{\sqrt{\sigma}}{\gamma}$.
  
  \item 
    Here $f_x = 2x - 4y$, $f_{xx} = 2$, $f_y = -4x + 3y^2 + 4$, and $f_{yy} = 6y$, so
    $f_{xx} + f_{yy} = 2 + 6y$.
  
  \item
    From given equation, $\frac{dy}{1+y^2} = \frac{dx}{1+x^2}$.
    Let $\theta_y = \tan^{-1} y$ and $\theta_x = \tan^{-1} x$,
    then an indefinite integral of $\frac{dy}{1+y^2}$ is $\theta_y$ and
    that of $\frac{dx}{1+x^2}$ is $\theta_x$
    Hence, using some integral constant $C_0$, it holds
    $\theta_y = \theta_x + C_0$ and
    \begin{align*}
      y &= \tan(\theta_x + C_0) \\
        &= \frac{\tan \theta_x + \tan C_0}
          {1 - \tan{\theta_x} \tan{C_0}} \\
        &= \frac{x + C}{1 - Cx},
    \end{align*}
    where $C = \tan C_0$.

  \item 
    From given equation, $\frac{dy}{1+y} = x dx$ and $\ln |1+y| = \frac{1}{2} x^2 + C$ by taking indefinite integration.
    Hence $y = A \exp \left( \frac{1}{2} x^2 \right) - 1$, where $A = \pm \exp C$.
  
  \item 
    Let $u_0 = y $ and $u_1 = y'$, then given equation is equivalent to
    $$
      \frac{d}{dx}
      \begin{pmatrix}
        u_1 \\
        u_0
      \end{pmatrix}
      =
      \begin{pmatrix}
        4 & -13 \\
        1 & 0
      \end{pmatrix}
      \begin{pmatrix}
        u_1 \\
        u_0
      \end{pmatrix}
    $$
    Hence the solution is 
    $$
      \begin{pmatrix}
        u_1 \\
        u_0
      \end{pmatrix}
      =
      \exp \left(
        x
        \begin{pmatrix}
          4 & -13 \\
          1 & 0
        \end{pmatrix}
      \right)
      \begin{pmatrix}
        a_1 \\
        a_0
      \end{pmatrix}
    $$
    for some initial value $(u_1(0), u_0(0)) = (a_1, a_0)$.
    We need to calulate the right hand side explicitly.
    Set $A = \begin{pmatrix}
      4 & -13 \\
      1 & 0
    \end{pmatrix}
    $. 

    First we calculate the eigencvalue of the matrix $A$.
    Let $E$ be the identity matrix.
    Solve $\det A - \lambda E = 0$,
    then $\lambda = 2 \pm 3 \sqrt{-1}$.
    
    $A$ has distinct eigenvalues, so $A$ is diagonalizable and
    $A = UDU^*$ for some unitary matrix and
    $$
      D = \begin{pmatrix}
        2 + 3 \sqrt{-1} & 0 \\
        0 & 2 - 3 \sqrt{-1}
      \end{pmatrix}.
    $$
    Hence
    \begin{align*}
      \exp(xA) &= U\exp(xD)U^* \\
        &= U \begin{pmatrix}
          e^{(2 + 3 \sqrt{-1})x} & 0 \\
          0 & e^{(2 - 3 \sqrt{-1})x}
        \end{pmatrix} U^* \\
        &= U \begin{pmatrix}
          e^{2x}(\cos 3x + \sqrt{-1} \sin 3x) & 0 \\
          0 & e^{2x}(\cos 3x - \sqrt{-1} \sin 3x)
        \end{pmatrix} U^*
      \end{align*}
    and
    $$
      \begin{pmatrix}
        u_1 \\
        u_0
      \end{pmatrix}
      =
      U \begin{pmatrix}
        e^{2x}(\cos 3x + \sqrt{-1} \sin 3x) & 0 \\
        0 & e^{2x}(\cos 3x - \sqrt{-1} \sin 3x)
      \end{pmatrix} U^*
      \begin{pmatrix}
        a_1 \\
        a_0
      \end{pmatrix}.
    $$

    Now $u_1$ and $u_0$ take real values if the initial value is real,
    so taking real part after matrix multiplicaion
    $$
      y = u_0 = e^{2x}(A \cos 3x + B \sin 3x).
    $$
    for some constants $A$ and $B$.

  \item 
    Consider $y = x^n u$, then
    \begin{align*}
      y' &= n x^{n-1} u + x^n u', \\
      y'' &= n(n-1) x^{n-2} u + 2n x^{n-1} u' + x^n u''
    \end{align*}
    and the ODE for $u$ is 
    $$
      \left( n(n-1) - 4n + 6 \right) x^n u + (2n - 4) x^{n+1} u' + x^{n+2} u'' = 0.
    $$

    We would like to choice $n$ such that the coefficient of $0$-th order in the ODE of $u$ is $0$.
    This coefficient is $n(n-1) - 4n +6 = (n-2)(n-3)$, so we choice $n=2$.
    
    Then the ODE is 
    $$
      x^{n+2} u'' = 0
    $$
    and the solution is $u = A + Bx$ for some constants $A$ and $B$.
    Hence the solution $y$ is
    $$
      y = x^2 u = A x^2 + B x^3.
    $$

  \item 
    From $\begin{vmatrix} k & k \\ 8 & 4k \end{vmatrix} = 4k^2 - 8k = 4k(k-2)$,
    the solution is $k = 0, 2$.

  \item 
    The augumented matrix of given linear system is
    $$
      \begin{pmatrix}
        2 & 1  & -1 & 1 \\
        1 & 0  & -2 & -5 \\
        1 & -2 & 3  & 6
      \end{pmatrix}.
    $$
    Using elementary row operation,
    \begin{align*}
      \begin{pmatrix}
        2 & 1  & -1 & 1 \\
        1 & 0  & -2 & -5 \\
        1 & -2 & 3  & 6
      \end{pmatrix}
      &\to
      \begin{pmatrix}
        0 & 1  & 3  & 11 \\
        1 & 0  & -2 & -5 \\
        0 & -2 & 5  & 11
      \end{pmatrix} \\
      &\to
      \begin{pmatrix}
      0 & 1  & 3  & 11 \\
      1 & 0  & -2 & -5 \\
      0 & 0  & 11 & 33
      \end{pmatrix} \\
      &\to
      \begin{pmatrix}
      0 & 1  & 3  & 11 \\
      1 & 0  & -2 & -5 \\
      0 & 0  & 1  & 3
      \end{pmatrix}.
    \end{align*}
    This means that $z = 3, y = -3z + 11 = 2, x = 2z - 5 = 1$.

  \item 
    The determinant of given matrix is $
      20 + 0 + (-9) - 18 - 0 - (-2) = -5
    $.

  \item
    The inner product $\mathbf{u} \cdot \mathbf{v} = -6 + 0 + 2k = 2k - 6$,
    so the value of $k$ such that $\mathbf{u} \cdot \mathbf{v} = 0$ is $k = 3$.

  \item
    $|x| = \sqrt{2^2 + (-3)^2} = \sqrt{13}$. 

  \item
    $f(\sin 2\theta, \cos 2\theta) = \sin^2 2\theta + \cos^2 2\theta = 1$,
    so $\frac{df}{d\theta} = 0$.

  \item 
    Solve $\begin{vmatrix} 2 - \lambda & 2 \\ 1 & 3 - \lambda \end{vmatrix}
    = (\lambda - 1)(\lambda - 4) = 0$, 
    then $\lambda = 1, 4$.

  \item
    $\int_{-2}^{1} |x| dx = \int_{0}^{2} x dx + \int_{0}^{1} x dx = \frac{1}{2} (2^2 + 1) =  \frac{5}{2}$.

  \end{enumerate}
 
  \rightline{End.}

\end{document}
